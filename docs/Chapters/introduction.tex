\chapter{Introduction}
\label{chap:introduction}
This is the documentation for the Bonsai workflow developed for the Sound Lateralization Task that is going to be performed by the Circuit Dynamics and Computation group  at the Champalimaud Foundation. The goal of this document is to complement the information present in the README of the \href{https://github.com/fchampalimaud/CDC.SoundLateralizationTask}{GitHub repository of the task}. 

The idea is that a user can easily download and setup the task just by reading the README, but that the implementation details of the task are also documented somewhere in case there is a need to understand them.

The experimental setup makes use of the capabilities of the Harp devices (which implement the \href{https://harp-tech.org/}{Harp} protocol) and the Bonsai visual programming language, which work really well together. 

A custom Bonsai package called \href{https://www.nuget.org/packages/CDC.SLTUtils/}{CDC.SLTUtils} was developed. This package adds new nodes with functionality which was difficulty to implement or wasn't available natively in Bonsai. The source code is also available in the task's \href{https://github.com/fchampalimaud/CDC.SoundLateralizationTask}{GitHub repository} in the \textit{./package} folder.